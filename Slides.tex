\documentclass{beamer}
\usepackage{amsmath}
%\usepackage{beamerthemesplit}
\usetheme{Goettingen}
\title{Essence of Functional programming}
\subtitle{Part 2}
\author{Clara B. Behrmann \and Johan Brinch  \and  Frej Soya}
\date{\today}
\newcommand{\bind}{\texttt{>>=}}
\newcommand{\ret}{\texttt{return}}
\newcommand{\bs}{\char`\\}
\newcommand{\fs}{\char`/}
\newcommand{\at}{\texttt{a}}
\newcommand{\kt}{\texttt{k}}
\newcommand{\mt}{\texttt{k}}
\begin{document}

\begin{frame}
  \titlepage
\end{frame}


\section*{Outline}
\begin{frame}
  \tableofcontents
\end{frame}

\section{Monads Invariants}
\subsection{The  3 Invariants}
\frame{
 \frametitle{Monad Invariants}
\begin{align}
\tag{Left identity}  &\ret\ \at~ & &\bind~\texttt{k}& &= \texttt{k a}\\ 
\tag{Right identity} &\mt~ & &\bind\ \ret&  &= \texttt{m} 	 \\
\tag{Associativity} &ToDO 
\end{align}
}
\subsection{Using the invariants}

\frame{
\frametitle{The Error monads }
Show how the error monad follows the invariants (or pick another monad)
}
\frame{
\frametitle{Monad syntax in haskell}
1-3 slides of examples comparing bind/unit with do-syntax (transliteration)
}
\frame{
\frametitle{How does them monad laws help me?}
1-3. How monads help the naive programmer. (Don't break normal expectations)
}
\subsection{Monads and lists}

\frame{
\frametitle{For list monads}
For the list monad together with the 3 invariants. The following invariants for Map and join apply.
\begin{itemize}
\item MapM 
\item JoinM
\item There are 7 extra invariants.
\end{itemize}
using the last one (8th) together with the 7 above, we can deduce the original 3 monad invariants 
}

\frame{
\frametitle{Monads can generalise list comprehensions}
Not that interesting?, Summary of another paper\\ 
Basicly any list comprehension in haskell can be translated to a monad.  \\
Thus list comphensions are syntactic sugar for some monads.
}
\section{Continuations}
\subsection{The CPS interpreter}
\frame{
\frametitle{Monad of continuations} 
The continuation monad ``\texttt{K}'' \\
Expressing continuations using monads. \\
TODO: Write the K monad in bind and return form.\\
}
\frame{
\frametitle{CPS interpreter} 
\emph{kill this slide}\\  
Creating a CPS intepreter using monads.
But this time wadlers also ``simplifies''
\begin{itemize}
\item By removing each of occurrence of \bind 
\item adding  ``bits'' to front to capture the continuation
\item adding ``bits to the end pass the continuation
\end{itemize}
 Each operation passes on the actual computation to the next function as a 'continuation'
}
\frame{
\frametitle{What does continuation mean?}
Example with add
Instead of evaluating the actual ``add``. The ``add'' computations is passed on to the next expression the interpreter meets. \\
This is done for all Terms, not just add.
}
\subsection{Call with current continuation}
\frame{
\frametitle{Call with current continuation}
This is too wordy on purpose... sorry about that ;). 
\begin{itemize}
\item Callcc takes the continuation and creates a new Function 
\item The function is saved in the enviroment
\item Interprets the remaining expression with new enviroment
\item Var extracts the continuation from the enviroment
\item The Function is called with the ``4'' as argument
\end{itemize}
}

\subsection{Monads can express CPS}
\frame{
\frametitle{The CPS interpreter can act as a Monad interpreter}
By selecting the right type for \texttt{Answer}, the CPS interpreter can act as the original monad based interpreter. \\
TODO: Write the definitions of PromoteK, ShowK\\
}
\frame{
\frametitle{Preserves modularity}
Wadler shows it's just as simple with the CPS style monad interpreter as using the monad interpreter. \\ 
Examples with Error, State and Output monad. (Not that simple but anyhow).
}
\subsection{CPS can express Monads}
\frame{
\frametitle{Comparing CPS and Monads}
Wadler compares CPS and Monads
\begin{itemize}
\item CPS always provides escape facility.
\item For monads it's a choice
\item We can fix this with expressing CPS as monads.
\end{itemize}
}

\section{What's the gain\ldots}
\subsection{I don't care about interpreters!}
\frame{
\frametitle{Applying this from a usage viewpoint}
If i'm not a language researcher, when is this cool?
\begin{itemize}
\item Example A
\item Example B
\item Example C
\end{itemize}
}
\subsection{Examples}
\frame{
\frametitle{Example A}
A
}
\frame{
\frametitle{Example B}
B
}
\frame{
\frametitle{Example C}
C
}
\end{document}
			    
