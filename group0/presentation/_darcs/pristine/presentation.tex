\documentclass{beamer}
\usepackage[T1]{fontenc}
\usepackage{textcomp}
\usepackage[utf8x]{inputenc}
\usepackage[danish]{babel}
\usepackage[garamond]{mathdesign}
\usepackage{url}
\usepackage{listings}
\usepackage{graphicx}
\usepackage{soul}

\renewcommand{\ttdefault}{pcr} % bedre typewriter font
\renewcommand{\rmdefault}{ugm} % garamond
\renewcommand{\sfdefault}{phv} % sans-serif font

\title{The essence of functional programming}
\subtitle{The paper by Philip Wadler}

\author{Martin Dybdal \and Troels Henriksen \and Ulrik Rasmussen}

\institute{\textrm{Datalogisk Institut, Københavns Universitet}}
\date{\today}

\mode<presentation>
{
  \usetheme{Frankfurt}
  %\usetheme{Warsaw} 
  \definecolor{uofsgreen}{rgb}{.125,.5,.25}
  \definecolor{natvidgreen}{rgb}{.196,.364,.239}
  \definecolor{kugrey}{rgb}{.4,.4,.4}
  \usecolortheme[named=uofsgreen]{structure}
  \usefonttheme[onlylarge]{structuresmallcapsserif}
  \usefonttheme[onlysmall]{structurebold}
}

\logo{\includegraphics[height=1.5cm]{diku.png}}

\usenavigationsymbolstemplate{} % fjern navigation

\lstset{language     = Haskell,
        extendedchars= true,
        breaklines   = false,
        tabsize      = 2,
        showstringspaces = false,
        basicstyle   = \small\ttfamily,
        commentstyle = \em,
        inputencoding= utf8
      }

\setcounter{tocdepth}{1}

\begin{document}

\frame{\titlepage}


\section{Introduction}
\subsection{Agenda}
\begin{frame}
  \frametitle{Agenda}
  \tableofcontents
\end{frame}

\subsection{What's a monad?}
\begin{frame}[fragile]
  \frametitle{What's a monad?}

  A triplet:
  \begin{itemize}
  \item<1-> A unary type constructor M
  \item<1-> A lifting function \texttt{unitM\footnote{This function is called \texttt{return} in
    Haskell and \textit{val} in Andrzej Filinski nomenclature.} :: a -> M a} that lifts a simple
    value into the monad. Creating a \textit{monadic
      value}. 
  \item<1-> A composition function \texttt{bindM\footnote{\texttt{>>=} in Haskell.} :: M a -> (a -> M b) -> M b}
     that applies a monadic function to a monadic value. 
  \end{itemize}
\pause
Obeying three laws (discussed later):
  \begin{enumerate}
  \item<2-> \lstinline{(unitM v) `bindM` f = f v}
  \item<2-> \lstinline{v `bindM` unitM = v}
  \item<2-> \begin{lstlisting}
(m `bindM` f) `bindM` g = 
           m `bindM` (\x -> (f x) `bindM` g)
          \end{lstlisting}
  \end{enumerate}
\end{frame}

\subsection{Running example: Interpreter}
\begin{frame}[t, fragile]
   \frametitle{Running example: Interpreter}
Datatypes:
\begin{lstlisting}
  type Name            =  String

  data Term            =  Var Name
                       |  Con Int
                       |  Add Term Term
                       |  Lam Name Term
                       |  App Term Term
  
  data Value           =  Wrong
                       |  Num Int
                       |  Fun (Value -> M Value)

  type Environment     =  [(Name, Value)]
\end{lstlisting}
\end{frame}

\begin{frame}[t, fragile]
   \frametitle{Running example: Interpreter}

\begin{lstlisting}
interp             :: Term -> Environment -> M Value
interp (Var x) e   = lookup x e
interp (Con i) e   = unitM (Num i)
interp (Add u v) e = interp u e `bindM` (\a ->
                     interp v e `bindM` (\b ->
                     add a b))
interp (Lam x v) e = unitM 
                      (Fun (\a -> interp v ((x,a):e)))
interp (App t u) e = interp t e `bindM` (\f ->
                     interp u e `bindM` (\a ->
                     apply f a))
test               :: Term -> String
test t = showM (interp t [])
\end{lstlisting}
\end{frame}

\section{Error messages}
\subsection{Error messages}
\begin{frame}[t, fragile]
  \frametitle{Define new monad}

\begin{lstlisting}
data E a              = Success a | Error String

unitE a               = Success a
errorE s              = Error s

(Success a) `bindE` k = k a
(Error s) `bindE` k   = Error s

showE (Success a)     = ''Success: '' ++ showval a
showE (Error s)       = ''Error: '' ++ s
\end{lstlisting}

We modify the interpreter to use this monad.

\end{frame}

\begin{frame}[t, fragile]
  \frametitle{Modify interpreter}
\begin{lstlisting}
lookup x [] = errorE("unbound variable: "   ++ x)
add a b     = errorE("should be numbers: "  ++ showval a
                                     ++ "," ++ showval b)
apply f a   = errorE("should be function: " ++ showval f)
\end{lstlisting}

\texttt{test term0} $\rightarrow$ "Success: 42"

\texttt{test (App (Con 1) (Con 2))} $\rightarrow$ "Error: should be function: 1"

\end{frame}

\subsection{Error messages with position}
\begin{frame}[t, fragile]
  \frametitle{Define monad based on the \texttt{E} monad}
  \begin{lstlisting}
    data Term = ... | At Position Term
    
    type  P a   = Position -> E a
    unitP a     = \p -> unitE a
    errorP s    = \p -> errorE (showpos p ++ ": " ++ s)
    m `bindP` k = \p -> m p `bindE` (\x -> k x p)
    showP m     = showE (m pos0)

    resetP     :: Position -> P x -> P x
    resetP q m = \p -> m q

    interp (At p t) e = resetP p (interp t e)
  \end{lstlisting}
  
  \subsubsection{The benefit of monads}
  \begin{itemize}
    \item Special control flow implicit, not explicit.
    \item Easy to extend monadic program.
    \item Cleanly separates different parts of program logic.
  \end{itemize}
\end{frame}


\section{Output}
\begin{frame}[t, fragile]
  \frametitle{The Output monad}
  \begin{itemize}
  \item<1-> Type constructor: \\
    \begin{lstlisting}
      type O a = (String, a)
    \end{lstlisting}
  \item<1-> Lifting: 
    \begin{lstlisting}
      unitO :: a -> O a
      unitO a = ("", a)
    \end{lstlisting}
  \item<1-> Composition:
    \begin{lstlisting}
      bindS :: O a -> (a -> O b) -> O b
      m `bindS` k = let (r, a) = m
                        (s, b) = k a
                    in (r ++ s, b)
    \end{lstlisting}
  \end{itemize}  

Writing output:

\begin{lstlisting}
  outO :: Value -> O ()
  outO a = (showval a ++ ";", ())
\end{lstlisting}

\end{frame}


\section{State}
\subsection{The State monad}
\begin{frame}[fragile]
  \frametitle{The State monad}
  \begin{itemize}
  \item<1-> Type constructor: \\
    \begin{lstlisting}
      type S a = State -> (a, State)
    \end{lstlisting}
  \item<1-> Lifting: 
    \begin{lstlisting}
      unitS :: a -> S a
      unitS a = \s -> (a, s)
    \end{lstlisting}
  \item<1-> Composition:
    \begin{lstlisting}
      bindS :: S a -> (a -> S b) -> S b
      m `bindS` k = \s0 -> let (a, s1) = m s0
                               (b, s2) = k a s1
                           in (b, s2)
    \end{lstlisting}
  \end{itemize}  

  Remark: \\
  We are not actually specifying the type of value our state should hold.
\end{frame}

\subsection{Example: Counting reductions}
\begin{frame}[t, fragile]
  \frametitle{Example: Counting reductions}

  \begin{columns}[t]
    \column<2->{.6\textwidth}

Our reduction counter is represented by an integer:
\begin{lstlisting}
  type State = Integer
\end{lstlisting}

"`Running"' the monad:
\begin{lstlisting}
  showS m = let (a, s1) = m 0
            in  "Value: "
             ++ showval a 
             ++ "; " 
             ++ "Count: " 
             ++ showint s1
\end{lstlisting}

Updating and fetching the state:
\begin{lstlisting}
  tickS   =  \s -> ((), s+1)
  fetchS  =  \s -> (s, s)
\end{lstlisting}
    \column<1->{.4\textwidth}
    \begin{block}{The State monad}
      
\color{kugrey}
    \begin{lstlisting}[basicstyle=\footnotesize\ttfamily]
type S a = 
  State -> (a, State)

unitS a = \s -> (a, s)

m `bindS` k = \s0 -> 
  let (a, s1) = m s0
      (b, s2) = k a s1
  in (b, s2)
    \end{lstlisting}
    \end{block}
  \end{columns}
\end{frame}

\begin{frame}[t, fragile]
  \frametitle{Example: Counting reductions}

  \begin{columns}[t]
    \column<1->{.6\textwidth}
\quad \\
\quad \\

Updating and fetching the state:
\begin{lstlisting}
  tickS   =  \s -> ((), s+1)
  fetchS  =  \s -> (s, s)
\end{lstlisting}
\quad \\
\quad \\

Doing the actual counting
\begin{lstlisting}
apply (Fun k) a = 
  tickS `bindS` (\() -> k a)

add (Num i) (Num j) = 
  tickS `bindS` (\() -> unitS (Num (i+j)))
\end{lstlisting}

\column<1->{.4\textwidth}
\begin{block}{The State monad}

\color{kugrey}
\begin{lstlisting}[basicstyle=\footnotesize\ttfamily]
type S a = 
  State -> (a, State)

unitS a = \s -> (a, s)

m `bindS` k = \s0 -> 
  let (a, s1) = m s0
      (b, s2) = k a s1
  in (b, s2)
\end{lstlisting}
\end{block}
\end{columns}
\end{frame}

\subsection{Backward state}
\begin{frame}[t, fragile]
\frametitle{Braintwister: Backward state}
\textbf{Warning:} This next example will hurt your brain!\\
\quad \\

We change the bind-operation from the State monad, so the
State-information flows backward:
    \begin{lstlisting}
      m `bindS` k = \s0 -> let (a, s1) = m s0
                               (b, s2) = k a s1
                           in (b, s2)
    \end{lstlisting}

becomes
    \begin{lstlisting}
      m `bindS` k = \s2 -> let (a, s0) = m s1
                               (b, s1) = k a s2
                           in (b, s0)
    \end{lstlisting}


\end{frame}

\begin{frame}[t, fragile]
  \frametitle{Computing the fibonacci sequence "`backwards"'}
  \begin{block}{Backward state bind}
    \begin{lstlisting}
      m `bindS` k = \s2 -> let (a, s0) = m s1
                               (b, s1) = k a s2
                           in (b, s0)
    \end{lstlisting}
  \end{block}
\pause
  \begin{lstlisting}
computeFibs = evalState [] 
      (fetchS `bindS` \fibs -> modify cumulativeSums 
              `bindS` \_ -> updateS (1:fibs) 
              `bindS` \_ -> unitS fibs)
\end{lstlisting}
\pause
\begin{lstlisting}
updateS   =  \a s -> ((), a)
evalState start s = snd (s [])
\end{lstlisting}
\pause
  \begin{lstlisting}
>> take 15 computeFibs
[0,1,1,2,3,5,8,13,21,34,55,89,144,233,377]
  \end{lstlisting}

  Found at {\footnotesize  http://lukepalmer.wordpress.com/2008/08/10/mindfuck-the-reverse-state-monad/}
\end{frame}

\section{Non-deterministic choice}
\subsection{Non-deterministic choice}
\begin{frame}[fragile]
 \frametitle{Non-deterministic choice}
 We modify the interpreter to deal with a non-deterministic language that
 returns a list of possible answers. We therefore need to define bind and
 return for lists:
 \begin{lstlisting}
  type L a = [a]

  unitL a     = [a]
  m `bindL` k = concat (map k m)
  zeroL       = []
  l `plusL` m = l ++ m
 \end{lstlisting}
\end{frame}

\begin{frame}[fragile]
 \frametitle{Non-deterministic choice, cont'd}
 Extend the interpreted language:
 \begin{lstlisting}
  data Term           = ... | Fail | Amb Term Term

  interp Fail e       = zeroL
  interp (Amb u v) e  = interp u e `plusL` interp v e
 \end{lstlisting}

 Now, interpreting the expression
 \begin{lstlisting}
  (App (Lam "x" (Add (Var "x") (Var "x"))) 
       (Amb (Con 1) (Con 2)))
 \end{lstlisting}

 returns \texttt{"[2,4]"}.
\end{frame}

\section{Call-by-name interpreter}
\subsection{Call-by-name}
\begin{frame}[fragile]
 \frametitle{Call-by-name}
 Modify the interpreter to call-by-name instead of call-by-value.
 Representations of functions should now be functions from computations to
 computations, and the environment should store computations instead of values:
 \begin{lstlisting}
  data Value       = Wrong
                   | Num Int
                   | Fun (M Value -> M Value)

  type Environment = [(Name, M Value)]
 \end{lstlisting}
\end{frame}

\begin{frame}[fragile]
 \frametitle{Call-by-name, cont'd}
 Subtle modifications to the code is also required. When applying a value to a
 lambda expression, only the function is evaluated:
 \begin{lstlisting}
 interp (App t u) e  = interp t e `bindM` (\f ->
                       apply f (interp u e))
 \end{lstlisting}

 When looking up variables in the environment, we no longer need to lift the
 values into the monad, as they are already computations:
 \begin{lstlisting}
 lookup x []        = unitM Wrong
 lookup x ((y,n):e) = if x==y then n else lookup x e
 \end{lstlisting}
\end{frame}

\begin{frame}[fragile]
 \frametitle{Call-by-name, cont'd}
 Example: If implemented for a non-deterministic language, interpreting the
 expression
 \begin{lstlisting}
  (App (Lam "x" (Add (Var "x") (Var "x")))
       (Amb (Con 1) (Con 2)))
 \end{lstlisting}

 Now returns \texttt{"[2,3,3,4]"}.
\end{frame}

\end{document}
